% %
% main.tex
%

% notes = hide | show | only
\documentclass[xcolor=dvipsnames,dvip,notes=show,table]{beamer}

% Para crear una versión 'handout' (impresa)
%\documentclass[xcolor=pst,dvips,handout,notes=show]{beamer}

\input{headers}



%%%%%%%%%%%%%%%%%%%%%%%%%%%%%%%%%%%%%%%%%%%%%%%%%%%%%%%%%%%%%%%%%%%%%%

\title[The RDFIndex approach | MTSR 2013]{Leveraging semantics to represent and compute quantitative indexes. \\ The RDFIndex approach.}
\author[Jose María Álvarez Rodríguez]{\textbf{Michalis Vafopoulos} (speaker) \\ and \\ Jose María Álvarez-Rodríguez}
\institute{MTSR 2013 | 7th Metadata and Semantics Research Conference \\ Main Track}


\date{}

\begin{document}

\frame{
\titlepage

}

\section{Introduction}

\frame{
  \frametitle{The Motivating example...} 
  
  
\begin{columns}[c] % the "c" option specifies center vertical alignment
\column{.5\textwidth} % column designated by a command


\begin{block}{Let's suppose that...}
 \begin{enumerate}
\item We want to create a ``Health index''...
\item ...to know which is the most ``healthy'' country.
\item ...to know which is the performance of the health expenditure .
\item ...to collect in just one value a set of indicators.
\item ...to make some new policy.
\end{enumerate}
\end{block}


\column{.5\textwidth}


\begin{figure}[htb]
\centering
	\includegraphics[width=4cm]{imgs/health}
%\caption{Modelo $5\star$ (W3C).}
\end{figure}

\end{columns}


}

\frame{
  \frametitle{The Motivating example...} 
  

\begin{block}{It is not a big deal because...}
 \begin{itemize}
\item We have the \textbf{``Health Index''} by an international network of physicians and researchers (\url{http://www.healthindex.com/}).
\item ...or the \textbf{``Health Index''} by the United Nations (\url{http://hdrstats.undp.org/en/indicators/72206.html}).
\item ...or the \textbf{``Ocean Health Index''} by an international collaborative effort (\url{http://www.oceanhealthindex.org/}).
\item ...or the indicators in the \textbf{World Bank} (\url{http://data.worldbank.org/topic/health}).
\item \ldots
\end{itemize}
\end{block}

}



\frame{
  \frametitle{The Motivating example...} 
 
   
\begin{exampleblock}{Benefits of using an index...}
 \begin{enumerate}
\item Creation of valuable data and information.
\item Generation of know-how to make some policy.
\item Re-use of a great effort and commitment by experts in some area.
\item Rank entities according to a quantitative value.
\item \ldots
\end{enumerate}
\end{exampleblock}

}

\frame{
  \frametitle{The Motivating example...} 
 
\scriptsize
\begin{alertblock}{Drawbacks of existing indexes...}<1->
 \begin{enumerate}
\item \textbf{Data heterogenity}: different datasources, formats and access protocols.
\item \textbf{Structure}: math models to aggregate some indicators that can change over time.
\item \textbf{Computation process}: observations are gathered and processed, \textit{somehow}, generating a final value.
\item \textbf{Documentation}: mutilingual and multicultural character of information.
\item \ldots
\end{enumerate}
\end{alertblock}

\begin{exampleblock}{..that imply the \textbf{necessity} of...}<2->
 \begin{enumerate}
\item \textbf{Accessing} \textbf{data} and information under a \textbf{common and shared data model}.
\item \textbf{Representing} the evolving \textbf{structure of the index}.
\item \textbf{Computing} the index to improve transparency.
\item Providing \textbf{context-aware documentation}: user-profile.
\item ... \textbf{Exploiting valuable data and metadata}.
\end{enumerate}
\end{exampleblock}

}


\frame{
  \frametitle{...but...Is it a common problem?} 
  

\begin{columns}[c] % the "c" option specifies center vertical alignment
\column{.5\textwidth} % column designated by a command


\begin{exampleblock}{...some indexes (per domain)...}
 \begin{enumerate}
\item Bibliography: the JCR and JSR, etc.
\item Government: the GDP, etc.
\item Web: the Webindex, etc.
\item Health: (the aforementioned ones).
\item Cloud: the CSC Cloud Usage Index, the VMWare index, the SMI index, etc.
\item ...to name a few per domain and creators.
\end{enumerate}
\end{exampleblock}


\column{.5\textwidth}


\begin{figure}[htb]
\centering
	\includegraphics[width=5cm]{imgs/indexes}
%\caption{Modelo $5\star$ (W3C).}
\end{figure}

\end{columns}


}


\section{Related Work}
\frame{
  \frametitle{Statistics and the Web of Data} 
  \scriptsize
\begin{block}{Vocabularies}<1->
  \begin{enumerate}
 \item The Statistical Core Vocabulary~\cite{scovo} (SCOVO), a former standard to describe statistical information in the Web of Data (2009).
 \item The RDF Data Cube Vocabulary~\cite{rdf-data-cube}, an adaptation of the ISO standard SDMX (Statistical Data and Metadata Exchange Vocabulary) (2013). 
 \item ``DDI-RDF discovery vocabulary. a metadata vocabulary for documenting research and survey data.'' (2013)~\cite{DDI2013} .
\end{enumerate}
\end{block}

\begin{exampleblock}{Statistics and Linked Data}<2->
  \begin{enumerate}
 \item ``Defining and Executing Assessment Tests on Linked Data for Statistical Analysis'' (2011)~\cite{DBLP:conf/semweb/ZapilkoM11}.
 \item ``Publishing Statistical Data on the Web'' (2012)~\cite{DBLP:journals/ijsc/SalasMBCMA12}.
 \item ``Publishing open statistical data: the Spanish census'' (2011)~\cite{DBLP:conf/dgo/FernandezMG11}.
 \item ``Publishing Statistical Data following the Linked Open Data Principles: The Web Index Project'' (2012)~\cite{webindexlod}.
 \item ``Linked Open Data Statistics: Collection and Exploitation'' (2013)~~\cite{DBLP:conf/kesw/ErmilovMLA13}.
 \item Some works in the ``RDF Validation Workshop 2013'' (\url{http://www.w3.org/2012/12/rdf-val/}).
\end{enumerate}
\end{exampleblock}

}

\section{Related Work}
\frame{
  \frametitle{Preliminary Evaluation} 
  
 \begin{block}{Vocabularies...}<1->
  Existing RDF-based vocabularies enable us the possibility of modelling and representing statistical data.
 \end{block}

 
 \begin{alertblock}{Existing approaches...}<2->
  All of the approaches are/were focused on data publishing/consumption...but...
  \begin{itemize}
   \item \textbf{Validation} of statistical data and/or structure...and
   \item the \textbf{Computation} process are still \textbf{open issues}.
  \end{itemize}

 \end{alertblock}


}

\frame{
  \frametitle{Main Contributions} 
  
 \begin{exampleblock}{1-Representation}<1->
  A \textbf{high-level model on top of the RDF Data Cube Vocabulary} for representing quantitative indexes.
 \end{exampleblock}
 
 
 \begin{block}{2-Computation}<2->
  A \textbf{Java-SPARQL based processor} to exploit the meta-information, validate and compute the new index values.
 \end{block}

 }




\section{The RDFIndex approach}

\frame{
  \frametitle{Overview} 
}

\frame{
  \frametitle{Example} 
}

\frame{
  \frametitle{Step 1: FIXME} 
}


\section{Use Case: the PublicSpending initiative}

\frame{
  \frametitle{Overview} 
}

\frame{
  \frametitle{The CORFU technique in Action...} 
}


\section{Evaluation}

\frame{
  \frametitle{Design of the experiment} 
}



\frame{
  \frametitle{Results} 
}


\section{Discussion}

\frame{
  \frametitle{Advantages} 
}


\frame{
  \frametitle{Restrictions} 
}



\section{Conclusions and Future Work}

\frame{
  \frametitle{Conclusions} 
}


\frame{
  \frametitle{Future Work} 
}


\section{Metadata and Information}

\frame{
  \frametitle{Acknowledgements} 
\begin{figure}[!htb]
\centering
 \includegraphics[width=9cm]{imgs/linkedin}
\end{figure}
}


\frame{
  \frametitle{Contact} 
}

\frame{
  \frametitle{Credits} 
}

\frame{
  \frametitle{References} 
}


\frame{
  \frametitle{Thank you...} 
}


\frame{
\titlepage

}

\appendix
\section*{References}
\bibliographystyle{abbrv}
\bibliography{references}


% %%%%%%%%%%%%%%%%%%%%%%%%%%%%%%%%%%%%%%%%%%%%%%%%%%%%%%%%%%%%%%%%%%%%%%

\end{document}
