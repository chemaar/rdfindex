% KWEB Del 1.1 (WP1.1) Report
%

\documentclass[a4paper,twoside,12pt]{report}
\usepackage{times}
\usepackage{graphics}
%\usepackage{graphicx}
%\usepackage{epsfig}
%\usepackage[]{graphicx}
\usepackage{deliverable}
%\usepackage{pnamed}

%cabeceras

\usepackage[T1]{fontenc}
% españolización
\usepackage[english]{babel}
\usepackage[utf8]{inputenc}
\usepackage{rotating}
\usepackage{graphicx}
\usepackage{color}
\usepackage{xspace}
% mejoras visuales
\usepackage{enumerate}
\usepackage{fancyhdr}  % para configurar los encabezados
\usepackage{fancybox}  % para hacer cajitas
\usepackage[normal,oneline,sf,bf]{caption2}
\usepackage{titlesec}  % para configurar los títulos de sección
\usepackage{paralist}
\usepackage{url}
\usepackage{latexsym}
\usepackage{amsmath}
\usepackage{amssymb}
\usepackage{amsthm}

\usepackage{epigraph}

% citas, referencias e índices
\usepackage{cite}
%\usepackage{citesort}   % da errores al compilar
\usepackage{makeidx}

% incrustaciones de código fuente
%\usepackage[norules,nolineno]{lgrind}
\usepackage{verbatim}
\usepackage{listings}
%\usepackage{noweb,a4wide}


%\usepackage{textcomp}
\usepackage[right]{eurosym}

% columnas
\usepackage{multicol}

% tablas
\usepackage{longtable}

%\input{commands}

% \addto\captionsspanish{
%         \def\listtablename{\'Indice de tablas}%
%         \def\tablename{Tabla}} 



\lstset{%
        language=Java,
	basicstyle=\footnotesize\sffamily,
	keywordstyle=\bfseries, %\color{darkred}
 	stringstyle=\itshape, %\color{violet}
 	commentstyle=\itshape, %\color{blue}
 	showspaces=false,
 	showtabs=false,
 	showstringspaces=false,
 	frame=trBL,
        frameround=tttt,
        %backgroundcolor=\color{lightyellow},
 	extendedchars=true,
 	numbers=none,
        aboveskip=0.5cm,
        belowskip=0.5cm,
        xleftmargin=1cm,
        xrightmargin=1cm,
	breaklines=true,
        morekeywords={PREFIX,java,rdf,rdfs,url,dcterms,foaf,skos,gr,skosxl}
}
\definecolor{darkred}{rgb}{0.5, 0, 0}
\definecolor{violet}{rgb}{1, 0, 1}
\definecolor{lightyellow}{rgb}{1,1,0.8}


\newtheorem{theorem}{Theorem}[section]
\newtheorem{proposition}[theorem]{Proposición}
\newtheorem{lemma}[theorem]{Lema}
\newtheorem{definition}[theorem]{Definición}
\newtheorem{examples}[theorem]{Ejemplo}
\newtheorem{remarks}[theorem]{Remarks}
\newtheorem{corollary}[theorem]{Corolario}
\newtheorem{remark}[theorem]{Remark}
\newtheorem{example}[theorem]{Ejemplo}
\newtheorem{conjecture}[theorem]{Conjecture}
\newtheorem{note}[theorem]{Nota}



\begin{document}

\title{Quality Management in Service-based Systems and Cloud Applications}

\author{
{Jose María Alvarez-SEERC\\
jmalvarez@seerc.org \\
josem.alvarez@josemalvarez.es}}

\revised{September 30, 2013}

%\issuer{Vrije Universiteit Amsterdam}

\identifier{WP4-SEERC-ER-TR}

\version{v1.0}

\state{Final}

\project{WP4‐Quality Management and Business Model Innovation}

\distribution{protected}

\synopsis{\strut\\
The FP7 Marie Curie Initial Training Network ``RELATE''  \\
Technical Report.

Keywords: cloud computing, services quality, semantics
}

\coverpages

\setcounter{page}{0}

\pagenumbering{roman}

\include{historial}
\include{resumen}

\setlength{\parskip}{0.0cm}
\tableofcontents
\listoffigures
\listoftables
\setlength{\parskip}{1.0ex}

\newpage
\pagenumbering{arabic}


%%%%%%%%%%%%%%%%%%%%%%%%%%%%%%
%\pagestyle{kweb}

%\input{chapters/intro}
%\input{chapters/consideraciones-generales}
% \input{chapters/personas}
%\input{localidades}
%\input{documentos}
%\input{organizaciones}
%\input{roles}
%\input{eventos}
%\input{proyectos-ley}

\bibliographystyle{unsrt}
% \bibliography{AEN01-00}

\end{document}
