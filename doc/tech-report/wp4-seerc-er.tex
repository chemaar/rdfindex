% KWEB Del 1.1 (WP1.1) Report
%

\documentclass[a4paper,twoside,12pt]{report}
\usepackage{times}
\usepackage{graphics}
%\usepackage{graphicx}
%\usepackage{epsfig}
%\usepackage[]{graphicx}
\usepackage{deliverable}
%\usepackage{pnamed}

%cabeceras

\usepackage[T1]{fontenc}
% españolización
\usepackage[english]{babel}
\usepackage[utf8]{inputenc}
\usepackage{rotating}
\usepackage{graphicx}
\usepackage{color}
\usepackage{xspace}
% mejoras visuales
\usepackage{enumerate}
\usepackage{fancyhdr}  % para configurar los encabezados
\usepackage{fancybox}  % para hacer cajitas
\usepackage[normal,oneline,sf,bf]{caption2}
\usepackage{titlesec}  % para configurar los títulos de sección
\usepackage{paralist}
\usepackage{url}
\usepackage{latexsym}
\usepackage{amsmath}
\usepackage{amssymb}
\usepackage{amsthm}

\usepackage{epigraph}

% citas, referencias e índices
\usepackage{cite}
%\usepackage{citesort}   % da errores al compilar
\usepackage{makeidx}

% incrustaciones de código fuente
%\usepackage[norules,nolineno]{lgrind}
\usepackage{verbatim}
\usepackage{listings}
%\usepackage{noweb,a4wide}


%\usepackage{textcomp}
\usepackage[right]{eurosym}

% columnas
\usepackage{multicol}

% tablas
\usepackage{longtable}

%\input{commands}

% \addto\captionsspanish{
%         \def\listtablename{\'Indice de tablas}%
%         \def\tablename{Tabla}} 



\lstset{%
        language=Java,
	basicstyle=\footnotesize\sffamily,
	keywordstyle=\bfseries, %\color{darkred}
 	stringstyle=\itshape, %\color{violet}
 	commentstyle=\itshape, %\color{blue}
 	showspaces=false,
 	showtabs=false,
 	showstringspaces=false,
 	frame=trBL,
        frameround=tttt,
        %backgroundcolor=\color{lightyellow},
 	extendedchars=true,
 	numbers=none,
        aboveskip=0.5cm,
        belowskip=0.5cm,
        xleftmargin=1cm,
        xrightmargin=1cm,
	breaklines=true,
        morekeywords={PREFIX,java,rdf,rdfs,url,dcterms,foaf,skos,gr,skosxl}
}
\definecolor{darkred}{rgb}{0.5, 0, 0}
\definecolor{violet}{rgb}{1, 0, 1}
\definecolor{lightyellow}{rgb}{1,1,0.8}


\newtheorem{theorem}{Theorem}[section]
\newtheorem{proposition}[theorem]{Proposición}
\newtheorem{lemma}[theorem]{Lema}
\newtheorem{definition}[theorem]{Definición}
\newtheorem{examples}[theorem]{Ejemplo}
\newtheorem{remarks}[theorem]{Remarks}
\newtheorem{corollary}[theorem]{Corolario}
\newtheorem{remark}[theorem]{Remark}
\newtheorem{example}[theorem]{Ejemplo}
\newtheorem{conjecture}[theorem]{Conjecture}
\newtheorem{note}[theorem]{Nota}



\begin{document}

\title{Quality Management in Service-based Systems and Cloud Applications}

\author{
{Jose María Alvarez-Rodríguez (SEERC)\\
jmalvarez@seerc.org \\
josem.alvarez@josemalvarez.es}}

\revised{September 30, 2013}

\issuer{Jose María Alvarez-Rodríguez}

\identifier{WP4-SEERC-ER-TR}

\version{v1.0}

\state{Final}

\project{WP4 Quality Management and Business Model Innovation}

\distribution{protected}

\synopsis{\strut\\
% Cloud Computing and Service Oriented Architectures have seen a dramatic increase of the amount of applications, 
% services, management platforms, data, etc. gaining momentum the necessity of new complex methods 
% and techniques to deal with the vast heterogeneity of data sources or services. In this sense Quality of Service (QoS) 
% seeks for providing an intelligent environment of self-management components based on domain knowledge in which 
% cloud components can be optimized easing the transition to an advanced governance environment. On the other hand, 
% semantics and ontologies have emerged to afford a common and standard data model that eases the interoperability, integration and 
% monitoring of knowledge-based systems. Taking into account the necessity of an interoperable and intelligent system to manage QoS in 
% cloud-based Systems and the emerging application of semantics in different domains, this paper reviews the main approaches for 
% semantic-based QoS management as well as the principal methods, techniques and standards for processing and exploiting diverse data 
% providing advanced real-time monitoring services. A semantic-based framework for QoS management is also outlined taking advantage of 
% semantic technologies and distributed datastream processing techniques. Finally a discussion of existing efforts and challenges are 
% also provided to suggest future directions.\\

Technical Report.


Keywords: cloud computing, services quality, semantics
}

\coverpages

\setcounter{page}{0}

\pagenumbering{roman}

% \include{historial}
% \include{resumen}

\setlength{\parskip}{0.0cm}
\tableofcontents
\listoffigures
\listoftables
\setlength{\parskip}{1.0ex}

\newpage
\pagenumbering{arabic}


%%%%%%%%%%%%%%%%%%%%%%%%%%%%%%
% \pagestyle{kweb}
\input{chapters/intro}
\input{chapters/background}
%The compilation of key performance indicators (KPIs) in just one value is 
becoming a challenging task in certain domains to summarize data and information. 
In this context, policymakers are continuously gathering and analyzing statistical 
data with the aim of providing objective measures about a specific policy, activity, 
product or service and making some kind of decision. Nevertheless existing tools 
and techniques based on traditional processes are preventing a 
proper use of the new dynamic and data environment avoiding more timely, 
adaptable and flexible (on-demand) quantitative index creation. On the other hand, 
semantic-based technologies emerge to provide the adequate building blocks 
to represent domain-knowledge and process data in a flexible fashion 
using a common and shared data model. That is why a RDF vocabulary designed on 
the top of the RDF Data Cube Vocabulary to model quantitative indexes 
is introduced in this paper. Moreover a Java and SPARQL based processor 
of this vocabulary is also presented as a tool to exploit the index meta-data structure and automatically 
perform the computation process to populate new values. Finally some discussion, 
conclusions and future work are also outlined.

\section{Related Work}
The present work is focused in applying semantic web vocabularies and datasets to model 
quantitative indexes from both structural and computational points of view. Currently one 
of the mainstreams in the Semantic Web area is the Linked Data initiative which principles 
have been applied to different domains such as  e-Government, e-Health, Biomedicine, Education, Bibliography or Geography 
to name a few,  with the aim of solving existing problems of integration and interoperability 
among applications and create a proper knowledge environment under Web-based protocolos. 

In order to reach this major objective the publication of information and data under a common data model (RDF) and formats with 
a specific formal query language (SPARQL~\cite{Sparql11}) provide the required building blocks to turn the Web of documents 
into a real database of data~\cite{freebase}. As a consequence the popular diagram of the Linked Data Cloud~\cite{linked-data-cloud}, 
generated from metadata extracted from the Comprehensive Knowledge Archive Network (CKAN~\cite{ckan}) out, 
contains $337$ datasets, with more than $25$ billion RDF triples and $395$ million links in different  domains. 
Research works are focused in two main areas: 1) production/publishing~\cite{bizer07how} and 2) consumption of  Linked Data. 
In the first case data quality~\cite{bizer2007,wiqa,ld-quality,lodq,link-qa}, conformance~\cite{HoganUHCPD:2012:237}, 
provenance~\cite{w3c-prov,DBLP:conf/ipaw/HartigZ10}, trust~\cite{Carroll05namedgraphs}, description of datasets~\cite{void,Cyganiak08semanticsitemaps,ckanValidator} and 
entity reconciliation~\cite{Serimi,Maali_Cyganiak_2011} are becoming major objectives since a mass of amount data is already available~\cite{Triplify} 
through SPARQL endpoints deployed on the top of RDF repositories such as OpenLink Virtuoso or OWLim. 

On the other hand, consumption of Linked Data is being addressed to provide new ways of data visualization~\cite{DBLP:journals/semweb/DadzieR11,hoga-etal-2011-swse-JWS}, 
faceted browsing~\cite{Pietriga06fresnel,citeulike:8529753,Sparallax} and searching~\cite{hoga-etal-2011-swse-JWS}, processing~\cite{Harth:2011:SIP:1963192.1963318} and exploitation of data applying 
different approaches such as sensors~\cite{Jeung:2010:EMM:1850003.1850235,ontology-search} and techniques  such as distributed 
queries\cite{Hartig09executingsparql,Ankolekar07thetwo,sparqlOpt}, scalable reasoning process~\cite{DBLP:conf/semweb/UrbaniKOH09,HoganHarthPolleres2009,DBLP:conf/semweb/HoganPPD10}, 
annnotation of web pages~\cite{rdfa-primer} or information retrieval~\cite{Pound} to name a few.

In the particular case of statistical data, the RDF Data Cube Vocabulary~\cite{rdf-data-cube}
a W3C Working Draft document, is a shared effort to represent statistical data in RDF reusing parts (the cube model) 
of the Statistical Data and Metadata Exchange Vocabulary (SDMX)~\cite{sdmx}, an ISO standard 
for exchanging and sharing statistical data and metadata among organizations. The Data Cube vocabulary is a core 
foundation which supports extension vocabularies to enable publication of other aspects of statistical data flows or 
other multi-dimensional data sets. Previously, the Statistical Core Vocabulary~\cite{scovo} was the standard in fact to describe statistical information in the Web of Data.
Some works are also emerging to mainly publish statistical data following the concepts of the LOD initiative 
such as~\cite{DBLP:conf/semweb/ZapilkoM11,DBLP:journals/ijsc/SalasMBCMA12,DDI2013,DBLP:conf/dgo/FernandezMG11,webindexlod} among others.

\section{Theoretical modeling of a quantitative composite index}
This section outlines a model for representing quantitative indexes based on the aggregation 
of different components and indicators. Furthermore a computation process for those elements is 
also presented in order to specify the population of new observations.

Basically, a quantitative index is comprised of the aggregation of several component observations. In the same way, 
a component is also composed of the aggregation of indicators that keep real observations. From this initial definition 
some characteristics and assumptions can be found: 1) although observations can be directly mapped to an index or a component, they 
are usually computed applying a bottom-up approach from an indicator to a component and index. 2) An observation is 
a real numerical value extracted from some agent out under a certain context. Generally observations only takes one measure and are considered 
to be raw without any pre-processing technique. 3) Before aggregating observation values, componens and indexes can 
estimate missing values to finally normalize them in order to get a final quantitative value.

According to the aforementioned characteristics and assumptions an ``observable'' element (index, component or indicator) is a 
dataset of numerical observations under a specific context (dimensions and/or metadata) that can be directly extracted from external 
sources out or computed by some kind of OWA operator. 

\begin{definition}[Observation-$o$]\upshape
It is a tuple $\{v,m,s\}$, where $v$ is a numerical value for the measure $m$ with an status $s$ that belongs to 
only one dataset of observations $O$. 
\end{definition}


\begin{definition}[Dataset-$q$]\upshape
It is a tuple $\{O,m,D,A,T\}$ where $O$ is a set of observations for only one measure $m$ that is described under 
a set of dimensions $D$ and a set of annotations $A$. Additionally, some attributes can be defined in the set $T$ for structure enrichment. 
\end{definition}


\begin{definition}[Aggregated dataset-$aq$]\upshape
It is an aggregation of $n$ datasets $q_i$ (identified by $Q$) which set of observations $O$ is derivated by applying 
an OWA operator $p$ to the observations $O_{q_i}$. 
\end{definition}

As a necessary condition for the computation process, an aggregated dataset $aq$ defined by means of the set of dimensions $D_{aq}$ can be computed iif 
$\forall q_j \in Q: D_{aq} \subseteq D_{q_j}$. Furthermore the OWA operator $p$ can only aggregate values belonging to the same measure $m$. Finally, 

As a consequence of the aforementioned definitions some remarks must be outlined in order to restrict the understanding of 
a quantitative index (structure and computation):
\begin{itemize}
 \item The set of dimensions $D$, annotations $A$ and $T$ for a given dataset $Q$ is always the same with the aim of describing all observations under 
 the same context.
 \item An index $i$ and a component $c$ are aggregated datasets. Neverthless this restriction is relaxed if observations can be directly mapped to 
 these elements without any computation processes.
 \item An indicator $in$ can be both dataset or aggregated dataset.
 \item All elements in definitions must be uniquely identified. 
 \item An aggregated dataset is also a dataset.
\end{itemize}

Following the on-going example, see Table~\ref{tab:example-wb}, the modelling of the ``The Naive World Bank Index'' would be the next one:
\begin{itemize}
 \item Each row of the table is an observation $o_i$ with a numerical value $v$, the measure is $m_{in}$ and the status is ``Raw''.
 \item Two indicators can be found: \{ ($in_1$, ``Life Expectancy''), ($in_2$, ``Health expenditure, total (\% of GDP)'') \}, each indicator contains a set 
 of observations $O_{in_i}$. The dimensions for each indicator are: $D_{in_1}$  \{(``Year'', ``Country'', ``Sex''\} and $D_{in_2}$ \{``Year'', ``Country''\}.
 \item In order to group the ``Life Expectancy'' without the ``Sex'' dimension it is necessary to define a new aggregated dataset $aq_1$ which 
 dimensions $D_{aq_1}$ are \{``Year'', ``Country''\} and the OWA operator is the average of values $v \in O_{in_1}$. In this sample the aggregated indicator $aq_1$
 can be built due to the indicator ``Life Expectancy'' accomplishes with the aforementioned necessary conditions: 1) $D_{aq} \subseteq D_{in_1} \wedge D_{aq_1} \subseteq D_{in_2}$ and 
 2) $m_{aq_1}= m_{in_1}$ = ``Life Expectancy''.
 \item In the same way, the set of components: \{($c_1$,``Aid Efectiveness''), ($c_2$,``Health'')\} are built aggregating the indicators $aq_1$ and 
 $q_2$ using as OWA operator the ``min'' value. In this case ``min'' or ``max'' operators can be used due to an observation is uniquely identified in a 
 dataset by a tuple $\{v,m,s\} \cup D$.
 \item Finally, the index is computed using the general form of an OWA operator $\sum_{i=1}^n  w_i c_i$ and taking as weights those we select.
\end{itemize}

As final remark, the computation process is generating new observations, following a bottom-up approach, according to the structure defined 
in each dataset. Although a logical structure of indexes, components and indicators can be directly established using narrower/broader properties 
the main advantage lies in the possibility of expressing new elements by aggregating others describing their structure. Nevertheless restrictions 
about the type of dataset that can be aggregated in each level could be added at any time for other reseasons such as validation or to generate 
a human-readble form of the index.


\section{Representation of a quantitative composite index in RDF: The RDFIndex}
Since previous section has stated the building blocks to represent quantitative indexes by aggregation a direct translation built 
on top of the RDF Data Cube Vocabulary, SDMX and other semantic web vocabularies is presented in Table~\ref{index-to-rdf}. Thus 
all concepts in the index are described reusing existing definitions, taking advantage of previous efforts and pre-established semantics 
with the aim of being extended in the future to fit new requirements. According to these mapping a definition of the index in the 
on-going example and some dimensions are presented in Figure~\ref{fig:results-rdf-index} and Figure~\ref{fig:results-rdf-index}.
FIXME: Maybe an observation?

\begin{table}[!htb]
\renewcommand{\arraystretch}{1.3}
\begin{center}
\begin{tabular}{|p{3cm}||p{6cm}|p{3cm}|}
\hline
  \textbf{Concept} & \textbf{Vocabulary element} &  \textbf{Comments}  \\  \hline
   Observation $o$ & \texttt{qb:Observation} &  \\ \hline
   Numerical value $v$ & \texttt{xsd:double} &  \\ \hline
   Measure $m$ & \texttt{qb:MeasureProperty} \texttt{sdmx-measure:obsValue} &  \\ \hline
   Status $s$ & \texttt{sdmx-concept:obsStatus} &  \\ \hline
   Dataset $q$ & \texttt{qb:dataSet} and \texttt{qb:qb:DataStructureDefinition} &  \\ \hline
   Dimension $d_i \in D$ & \texttt{qb:DimensionProperty} &  \\ \hline
   Annotation $a_i \in A$ & \texttt{owl:AnnotationProperty} &  FIXME: dublin core?\\ \hline
   Attribute $at_i \in T$ & \texttt{qb:AttributeProperty} &  \\ \hline
   OWA operator $p$ &  \texttt{skos:Concept} and SPARQL 1.1 aggregation operators & FIXME \\ \hline
   Index, Component and Indicator & \texttt{skos:Concept} & FIXME \\ \hline
  \hline
  \end{tabular}
  \caption{Mapping between the index definition and the RDF Data Cube Vocabulary.}
  \label{index-to-rdf}
  \end{center}
\end{table} 

\begin{figure}[!ht]
\begin{lstlisting}[language=XML]  
@prefix rdfindex:  <http://purl.org/rdfindex/ontology/> .
@prefix rdfindex-wb:  <http://purl.org/rdfindex/wb/resource/> .
@prefix rdfindex-wbont:  <http://purl.org/rdfindex/wb/ontology/> .

rdfindex-wb:TheWorldBankNaiveIndex 
  a rdfindex:Index;
  rdfs:label "The Weight Longest Life Country"@en;
  rdfindex:type rdfindex:Quantitative;
  rdfindex:aggregates [ 		
    rdfindex:aggregation-operator rdfindex:OWA;
    rdfindex:part-of [
      rdfindex:dataset rdfindex-wb:AidEffectiveness; 
      rdfindex:weight 0.4];
    rdfindex:part-of [rdfindex:dataset rdfindex-wb:Health; 
      rdfindex:weight 0.6];
  ];
  #More metadata properties...
  qb:structure 	rdfindex-wb:TheWorldBankNaiveIndexDSD ; .
  
rdfindex-wb:TheWorldBankNaiveIndexDSD 
  a qb:DataStructureDefinition;  
  qb:component    
  [qb:dimension rdfindex-wbont:ref-area],
  [qb:dimension rdfindex-wbont:ref-year],
  [qb:measure   rdfindex:value],
  [qb:attribute sdmx-attribute:unitMeasure];
  #More metadata properties...
  .
\end{lstlisting}
\caption{Example of an index structure in RDF.}
 \label{fig:results-rdf-index}
\end{figure}

\begin{figure}[!ht]
\begin{lstlisting}[language=SQL]  
rdfindex-wbont:ref-area a rdf:Property, 
  qb:DimensionProperty; 
   rdfs:subPropertyOf sdmx-dimension:ref-area; 
  rdfs:range skos:Concept; 
  qb:concept sdmx-concept:ref-area . 

rdfindex:value a rdf:Property, qb:MeasureProperty;
  rdfs:label "Value of an observation"@en;
  skos:notation "value" ;
  rdfs:subPropertyOf sdmx-measure:obsValue;
  rdfs:range xsd:double . 
\end{lstlisting}
\caption{Example of a dimension and a measure definition in RDF.}
 \label{fig:results-rdf-properties}
\end{figure}


Once the structure and the computation processes can be built on the top of existing RDF vocabulary it is also 
possible to make a translation to a generic SPARQL query (includes the basic OWA operator), see Figure~\ref{fig:results-rdf-sparql-template}, in order generate new observations following the bottom-up approach that previous section has presented.

\begin{figure}[!ht]
\begin{lstlisting}[language=SQL,mathescape]  
SELECT ($d_i \in D$) [(sum(?w*?measure) as ?newvalue) | OWA(?measure)]
WHERE{
  $q$ rdfindex:aggregates ?parts.
  ?parts rdfindex:part-of ?partof.
  ?partof rdfindex:dataset $q_i$ .
  FILTER($?partof \in Q$).  
  ?observation rdf:type qb:Observation.
  ?part rdfindex:weight ?defaultw.     
  OPTIONAL {?partof rdfindex:weight ?aggregationw.}.
  BIND (if( BOUND(?aggregationw), ?aggregationw, ?defaultw) AS ?w)
  ?observation $m$ ?measure . 
  ?observation ?dim ?dimRef. 
  FILTER ($?dim  \in D$).
}GROUP BY ($d_i \in D$)
\end{lstlisting}
\caption{SPARQL template for building aggregated observations.}
 \label{fig:results-rdf-sparql-template}
\end{figure}


\subsection{A $\{Java\rightarrow based \rightarrow SPARQL\}$  interpreter of the RDFIndex}
The first implementation of the RDFIndex vocabulary processor is based on traditional 
language processor techniques FIXME such as the use of design patterns (e.g. Visitor) 
to separate the exploitation of metadata from the interpretation. Thus the processor 
works and provides next functionalities (hereafter load and query an endpoint are completely the same due to 
data is separated from access and storage formats):

\begin{itemize}
 \item The RDFIndex ontology is loaded to have access to common definitions.
 \item The structure of an index described with the aforementioned vocabulary is 
 also loaded to create a kind of Abstract Syntax Tree (AST) containing the defined metadata.
 \item Once the metadata structure is available in the AST it can be leverage 
 through three AST walkers that performs: 1) validation (structure and RDF Data Cube normalization) 
 and 2) SPARQL queries creation and 3) documentation generation (optional). As an example a partial creation of a SPARQL implementing 
 the z-score normalization function is presented in Figure~\ref{fig:sparql-zscore}.
 \item In order to promote new observations to the different components and indexes 
 a set of raw observations is also loaded and a new AST walker generates new values, through SPARQL queries (see Figure~\ref{fig:sparql-generated-query}), 
 in a bottom-up approach until reaching the upper-level (index).
 \end{itemize}

\begin{figure}[!ht]
\begin{lstlisting}[language=SQL]  
prefix afn: <http://jena.hpl.hp.com/ARQ/function#>
SELECT ( (?measure-?mean)/?stddev as ?zscore) 
WHERE{
 ...
 ?observation rdfindex:value ?measure 
 {
  SELECT ?mean (afn:sqrt((SUM((?measure-?mean)*(?measure-?mean))/?count)) 
		as ?stddev) 
  WHERE{ 
   ?observation rdfindex:value ?measure 
   {
    SELECT (COUNT(?measure) as ?count) (AVG(?measure) as ?mean)
    WHERE {
     ?observation rdfindex:value ?measure 
     }GROUP BY ?count ?mean LIMIT 1

   }	
  }GROUP BY ?mean ?count LIMIT 1
 }
}
\end{lstlisting}
\caption{Z-score normalization using SPARQL.}
 \label{fig:sparql-zscore}
\end{figure}


% 
\begin{figure}[!ht]
\begin{lstlisting}[language=SQL]  
prefix rdfindex:  <http://purl.org/rdfindex/ontology/> 
SELECT ?dim0 ?dim1 ( sum(?w*?measure) as ?newvalue) 
WHERE{ 
  rdfindex-wb:TheWorldBankNaiveIndex  
  rdfindex:aggregates ?parts.
  ?parts rdfindex:part-of ?partof.
  ?partof rdfindex:dataset ?part .
  FILTER ((?part =rdfindex-wb:AidEffectiveness) || 
	  (?part =rdfindex-wb:Health)). 
  ?observation qb:dataSet ?part . 
  ?part rdfindex:weight ?defaultw.        
  OPTIONAL {?partof rdfindex:weight ?aggregationw.}.
  BIND (if( BOUND(?aggregationw), ?aggregationw, ?defaultw) AS ?w)
  ?observation rdfindex:value ?measure . 
  ?observation rdfindex-wbont:ref-area ?dim0. 
  ?observation rdfindex-wbont:ref-year ?dim1. 
} GROUP BY ?dim0 ?dim1 
\end{lstlisting}
\caption{Example of generated SPARQL query.}
 \label{fig:sparql-generated-query}
\end{figure}





\bibliographystyle{unsrt}
\bibliography{bib/cloud-references,bib/linked-data,bib/references}

\end{document}
